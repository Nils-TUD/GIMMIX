% option draft um zu lange Zeilen anzuzeigen
\documentclass[a4paper,11pt]{article}

\usepackage[english]{babel}
\usepackage[utf8]{inputenc}
\usepackage[T1]{fontenc}
\usepackage[inner=3.5cm,top=2cm,outer=3cm,bottom=3cm]{geometry}

% length of paragraph indent
\setlength{\parindent}{10pt}

% a few convenience commands
\renewcommand{\b}[1]{\textbf{#1}}
\renewcommand{\i}[1]{\emph{#1}}
\newcommand{\ie}{i.\,e. }
\newcommand{\eg}{e.\,g. }

\begin{document}

\pagenumbering{gobble}
\title{Development of a Simulator for the 64-Bit RISC\\ Processor MMIX}
\author{- Abstract -}
\date{Nils Asmussen}
\maketitle

MMIX (pronounced "em-mix") has been designed by Donald Knuth to be a successor of MIX -- the abstract machine that is used in the famous book series "The Art of Computer Programming" of Donald Knuth. Although MMIX is intended for educational purposes, he pursued the goal to achieve high performance in practice and that MMIX could in principle be built and be competitive with other computers in the marketplace.

\medskip

MMIX is a very complex architecture, that provides 256 instructions, 256 general purpose registers and 32 special registers. It offers a 64-bit virtual and physical address space with a complicated paging mechanism to translate virtual addresses to physical ones. Additionally, it uses a combination of registers and memory for the stack (called "register stack") for performance reasons. Last but not least, it supports both integer arithmetic and floating point arithmetic.

\medskip

The master thesis describes the architecture MMIX in general and the development of the simulator called GIMMIX (for "Gießen Implementation of MMIX"). The longterm goal of GIMMIX is to be able to port an operating system to MMIX. Although MMIXware does already provide two simulators for MMIX, written by Donald Knuth himself, it has been decided to develop a new one. The reason for this decision was, that the existing simulators were designed for a different purpose and are therefore not well suited for the mentioned ambition.

\medskip

For the development of GIMMIX, the programming language C99 has been chosen, because of the amount of control it offers and the performance that can be achieved. The most important goals of GIMMIX are to be correct, \ie respect the specification of the MMIX architecture, and to provide a convenient and productive user interface that allows it to debug an operating system. The former is ensured by a sophisticated test system, that tries to test every feature and special case of MMIX and compares the behaviour of GIMMIX with the behaviour of the two simulators published in MMIXware. For the latter, a command line interface has been developed that provides various commands for the user to debug an operating system (or other programs). It is based on a small language, which has been specifically designed for GIMMIX, to offer a lot of flexibility and a consistent user experience.

\medskip

The status of GIMMIX reached by the master project is, that the simulator is completely finished, \ie it realizes the entire MMIX architecture. Additionally, a convenient and powerful command line interface exists and the whole system has been tested as much as possible to reach the confidence, that everything works as intended. Therefore, the building blocks required to be able to port an operating system to MMIX are available.

\end{document}

