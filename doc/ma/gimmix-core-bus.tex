\subsection{Bus}

GIMMIX uses the module \i{bus} as an interface to the attached devices. The idea is to make the rest of the simulator core independent of the present devices. In this case, a compromise between simplicity and dynamic has been chosen. Because on the one hand, it is expected that some day more devices will be provided, but on the other hand, it is considered unlikely, that devices are offered by third party vendors. Thus, it is not that dynamic that no code change is necessary at all (by using shared libraries, for example), but dynamic enough to require as few code changes as possible and still keep the extension mechanism simple.

To achieve that, the bus maintains a single linked list with the attached devices. The init function of the bus, which is called during initialization of the simulator, calls the init function of all devices. These in turn will use `bus_register` to register themself to the bus. That is, they tell the bus their name, their address range in I/O space, their \glslink{Interrupt}{interrupt} mask and the callback functions for reading, writing, resetting the device and shutting it down. This way, for a new device, the file implementing the device has to be added and its init function has to be called in the init function of the bus. No other changes are necessary.

The most important functions of the bus are `bus_read` and `bus_write`. Since there are no interesting differences, it is sufficient to take a look at `bus_read`:
\begin{lstlisting}[language=C,caption={Implementation of {\tt bus\_read}}]
octa bus_read(octa addr,bool sideEffects) {
	addr &= ~(octa)(sizeof(octa) - 1);
	const sDevice *dev = bus_getDevByAddr(addr);
	if(dev == NULL)
		ex_throw(EX_DYNAMIC_TRAP,TRAP_NONEX_MEMORY);

	octa data = dev->read(addr);
	if(sideEffects)
		ev_fire2(EV_DEV_READ,addr,data);
	return data;
}
\end{lstlisting}
As the listing shows, at first the desired device is searched with `bus_getDevByAddr`. If no device is found, the physical memory will not exist and thus, an \glslink{Exception}{exception} will be thrown. Otherwise the read function of the device is called and the value is returned. It is noteworthy, that the main memory (RAM) and the ROM are devices as well, because the only difference to the actual I/O devices like terminal, disk and so on is the address range. That is, RAM uses \haddro{0000}{0000}{0000}{0000} to \haddro{0000}{FFFE}{FFFF}{FFFF} and ROM uses \haddro{0000}{FFFF}{0000}{0000} to \haddro{0000}{FFFF}{FFFF}{FFFF}, while the other devices use the space beginning at \haddro{0001}{0000}{0000}{0000}. To prevent the introduction of a special case for RAM and ROM, they are treated like all other devices as well.
