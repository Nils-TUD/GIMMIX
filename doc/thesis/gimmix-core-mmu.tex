\subsection{MMU}

The memory management unit is the first piece of the memory hierarchy in GIMMIX. It is responsible for loading values from the next piece of the hierarchy (the cache), performing address translation and mapping byte, wyde or tetra requests to octa requests, when accessing the cache.

The module \i{MMU} provides three kinds of functions for the simulator core:
\begin{enumerate}
	\item Functions to read from memory,
	\item functions to write to memory and
	\item functions to synchronize caches and memory.
\end{enumerate}
Additionally, the whole memory hierarchy uses four flags to communicate the desired behaviour to the different parts of the hierarchy:
\begin{enumerate}
	\item `MEM_SIDE_EFFECTS`: If disabled, the state of GIMMIX is not changed and no events are fired. This is used by the CLI, which should of course not change the state when for example a value is read from memory.
	\item `MEM_UNCACHED`: Tells the cache, that if a value is not yet in the cache, the affected cache block should not be loaded, but the value should be read/written directly from/to memory.
	\item `MEM_UNINITIALIZED`: If enabled, the cache will not load the affected cache block from memory, if not already present, but initialize the cache block to zero.
	\item `MEM_IGNORE_PROT_EX`: If enabled, no dynamic trap will be triggered for protection faults. This is used for synchronizing cache and memory, for which MMIX defines that no protection faults occur.
\end{enumerate}

\subsubsection{Reading from Memory}

The MMU module offers five reading functions -- one for each quantity and a separate function to read an instruction (to use the instruction TC and instruction cache). All read functions use the internal function `mmu_doRead` to perform the actual reading. The octa-version will simply return the value read by that function, while others extract the quantity from the corresponding position. For example, `mmu_readTetra` is implemented as:
\begin{lstlisting}[language=GIMMIXC,caption={Implementation of {\tt mmu\_readTetra}}]
tetra mmu_readTetra(octa addr,int flags) {
	int off = (addr & (sizeof(octa) - 1)) >> 2;
	octa data = mmu_doRead(addr,MEM_READ,flags);
	return (data >> (32 * (1 - off))) & 0xFFFFFFFF;
}
\end{lstlisting}
The actual reading function looks like the following:
\begin{lstlisting}[language=GIMMIXC,caption={Implementation of {\tt mmu\_doRead}}]
static octa mmu_doRead(octa addr,int mode,int flags) {
	octa res;
	jmp_buf env;
	int ex = setjmp(env);
	if(ex != EX_NONE) {
		mmu_handleMemEx(ex,addr,0,flags);
		// loads that cause an exception, load zero
		res = 0;
	}
	else {
		ex_push(&env);
		int exp = (mode & MEM_READ) ? MEM_READ : MEM_EXEC;
		int cache = (mode & MEM_READ) ? CACHE_DATA : CACHE_INSTR;
		octa phys = mmu_translate(addr,mode,exp,flags & MEM_SIDE_EFFECTS);
		res = cache_read(cache,phys,flags);
	}
	ex_pop();
	return res;
}
\end{lstlisting}
As the listing shows, \glslink{Exception}{exceptions} are catched here, because not all \glslink{Exception}{exceptions} actually cause a trap. It depends on the current \sr{K} and whether the flag `MEM_IGNORE_PROT_EX` is set. If no trap is caused, the instruction will have to be finished, \ie the register has to be set, for example. The decision whether to trap or not is made by `mmu_handleMemEx`, which uses `cpu_setMemEx` and calls `ex_rethrow` to throw the catched \glslink{Exception}{exception} again, if necessary. The first step of `mmu_doRead` is to translate the virtual address to a physical one via `mmu_translate`. Afterwards the octa is read from the cache, which in turn might request it from the corresponding device.

\subsubsection{Writing to Memory}

Before taking a closer look at the address translation, it should be described how the write functions work. Analogous to the read functions, the MMU provides a function for each quantity of MMIX, whereas all functions except the one that writes an octa, reads an octa first using `mmu_doRead`, replaces the corresponding byte, wyde or tetra and writes the octa back to memory using `mmu_doWrite`. This function is implemented as follows:
\begin{lstlisting}[language=GIMMIXC,caption={Implementation of {\tt mmu\_doWrite}}]
static void mmu_doWrite(octa addr,octa value,int flags) {
	jmp_buf env;
	int ex = setjmp(env);
	if(ex != EX_NONE) {
		mmu_handleMemEx(ex,addr,value,flags);
		// stores that cause an exception, store nothing
	}
	else {
		ex_push(&env);
		octa phys = mmu_translate(addr,MEM_WRITE,MEM_WRITE,flags & MEM_SIDE_EFFECTS);
		cache_write(CACHE_DATA,phys,value,flags);
	}
	ex_pop();
}
\end{lstlisting}
Of course, the implementation is very similar to the one of `mmu_doRead`. Exceptions are catched and handled by `mmu_handleMemEx`, the address is translated first and finally, it writes to the cache.

\subsubsection{Address Translation}

Virtual addresses are translated to physical ones via `mmu_translate`. Without going too far into the details here, it works roughly in the following way:
\begin{lstlisting}[language=GIMMIXC,caption={Implementation of {\tt mmu\_translate} (partially pseudo-code)}]
octa mmu_translate(octa addr,int mode,int expected,bool sideEffects) {
	int tc = (mode & MEM_EXEC) ? TC_INSTR : TC_DATA;
	if(sideEffects)
		ev_fire2(EV_VAT,addr,mode);
	if(addr & MSB(64))
		return addr & ~MSB(64);
	
	<check rV, i.e. page size and f field>
	octa pte;
	sTCEntry *tce = NULL;
	if(sideEffects)
		tce = tc_getByKey(tc,tc_addrToKey(addr));
	if(tce == NULL) {
		if(sideEffects && vtr.f == 1)
			ex_throw(EX_FORCED_TRAP,TRAP_SOFT_TRANS | mode);
		<translate the address, yielding the PTE>
		pte = tc_pteToTrans(pte);
		if(sideEffects)
			tc_set(tc,tc_addrToKey(addr),pte);
	}
	else
		pte = tce->translation;

	if(!(pte & expected))
		ex_throw(EX_DYNAMIC_TRAP,TRAP_REPEAT | (mode&~pte));
	octa trans = pte & ~(octa)0x7;
	octa phys = (trans << 10) + (addr & ((1 << vtr.s) - 1));
	return phys;
}
\end{lstlisting}
That means, if the address is in privileged space, no translation will be done, but only the most significant bit will be cleared. If it is in user space, the corresponding TC will be asked whether a translation exists. If so, it will be used, otherwise the actual address translation will be performed and the resulting PTE will be put into the translation cache. Additionally, if software translation is desired, the corresponding \glslink{Exception}{exception} will be thrown (`vtr` is a struct containing all fields of \sr{V}). As soon as the PTE is known, the access permissions are checked and if these are sufficient, the resulting physical address is returned.

It should be noted, that the function takes `mode` and `expected`, which are both a combination of the bits `MEM_READ`, `MEM_WRITE` and `MEM_EXEC`. The former determines the actual access type and the thrown \glslink{Exception}{exceptions}, whereas the latter contains the flags that are required to be set in the PTE. The reason for the separation is that the functions for writing a byte, wyde or tetra have to perform a read request first. Of course, this request should fail, if the page does not allow reads. But it may have to throw both a read and write protection fault, because we have to write afterwards. More precisely, the bits that are missing have to be reported. That is, if read permission is missing, but write permission is present, a read protection fault will be thrown. If both are missing, both will be thrown. For this reason, the read requests used in the write functions, use `MEM_READ` for `mode` and `MEM_READ | MEM_WRITE` for `expected`. In the ordinary cases, `mode` and `expected` are the same. Additionally, the way the functions for writing a byte, wyde or tetra to memory have to be implemented including the thrown \glslink{Exception}{exceptions} implies, that when a page is only writable, no bytes, wydes and tetras can be written to that page, but only octas.

Furthermore, the state will not be changed and no events will be fired, if no side effects are desired. In other words, if the CLI accesses memory, the translation will be done every time and the translation caches will be ignored.

\subsubsection{Translation Cache}

Since the translation procedure is handled in the MMU module, the module \i{TC} is very simple. It only holds two arrays with TC entries -- one for the instruction TC and one for the data TC -- and offers other modules access to it. The most important functions are `tc_getByKey`, `tc_set`, `tc_remove` and `tc_removeAll`. The first one searches for a given key (\ie more or less the virtual address) and returns the translation entry containing the key and the translation, if found. The function `tc_set` puts a translation into the TC, `tc_remove` removes a specific entry and `tc_removeAll` removes all entries from a TC. Last but not least, `tc_pteToTrans` and `tc_addrToKey` can be used to transform a PTE to a translation and a virtual address to a key, respectively.

