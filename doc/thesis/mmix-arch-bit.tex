\section{Bit Fiddling}

MMIX has quite a few instructions for manipulating bits. At first, the well known bit operations {\tt AND}, {\tt OR}, {\tt NOR}, \dots and shifts are described, because they will not be a surprise.

\subsection{Basic Bit Operations}

\instrtbl
	{\mi{AND|OR|XOR|ANDN|ORN|NAND|NOR|NXOR \$X,\$Y,\$Z|Z}}
	{$\dr{X} \leftarrow \dr{Y} \land|\lor|\oplus|\land \sim|\lor \sim|\mathbin{\overline\land}|\mathbin{\overline\lor}|\mathbin{\overline\oplus} \udrim{Z}$}
\noindent These instructions set \dr{X} to the result of the corresponding bit operation with operands \dr{Y} and \udrim{Z}. The instructions that end with '\mi{N}' logically negate \udrim{Z} first and apply the operation without '\mi{N}' (\mi{OR} or \mi{AND}) afterwards. \citep[pg. 7]{mmix-doc}

\instrtbl
	{\mi{SL|SLU|SR|SRU \$X,\$Y,\$Z|Z}}
	{$\dr{X} \leftarrow \dr{Y} \ll|\ll|\gg|\ggg \udrim{Z}$}
\noindent The instructions for shifting left, \mi{SL} and \mi{SLU}, have the same behaviour, except that \mi{SL} will raise an integer overflow \glslink{Exception}{AE}, if the result is $\ge 2^{63}$ or $< -2^{63}$. \mi{SR} performs an arithmetic right shift, \ie it shifts in copies of the sign bit from the left, and \mi{SRU} performs a logical right shift, \ie it shifts in zeros from the left. Since \udrim{Z} is treaten unsigned, one can not use \mi{SL} to shift right or similar. Additionally it is worth mentioning, that a logical left or right shift of 64 or more will set \dr{X} to zero, whereas an arithmetic right shift of 64 or more will set \dr{X} to $-1$, if \dr{Y} is negative and to zero otherwise. \citep[pg. 10]{mmix-doc}

\subsection{Wyde Operations}

If one liked to put an arbitrary 64-bit constant into a register or manipulate individual wydes of a registers, one could use one of the following 16 instructions.

\instrtbl
	{\mi{SETH|SETMH|SETML|SETL \$X,YZ}}
	{$\dr{X} \leftarrow {\tt YZ} \ll 48|32|16|0$}
\noindent These instructions set the corresponding wyde of \dr{X} to the 16-bit constant {\tt YZ} and all other wydes to zero \citep[pg. 7]{mmix-doc}. That means, for example \mi{SETMH \$X,\haddr{1234}} sets \dr{X} to \haddro{0000}{1234}{0000}{0000}.

\instrtbl
	{\mi{ORH|ORMH|ORML|ORL \$X,YZ}}
	{$\dr{X} \leftarrow \dr{X} \lor ({\tt YZ} \ll 48|32|16|0)$}
\noindent Similarly, these instructions {\tt OR} the 16-bit constant {\tt YZ} into the corresponding wyde of \dr{X} \citep[pg. 7]{mmix-doc}. For example, if \dr{X} is \haddro{0000}{F0F0}{FF00}{0000}, \mi{ORML \$X,\haddr{0FF0}} will result in \haddro{0000}{F0F0}{FFF0}{0000}.

\instrtbl
	{\mi{ANDNH|ANDNMH|ANDNML|ANDNL \$X,YZ}}
	{$\dr{X} \leftarrow \dr{X}~\land \sim({\tt YZ} \ll 48|32|16|0)$}
\noindent Analogous to the {\tt ORX} family, these instructions remove the bits set in the 16-bit constant {\tt YZ} from the corresponding wyde of \dr{X} \citep[pg. 7]{mmix-doc}. For example, if \dr{X} is \haddro{0000}{F0F0}{FF00}{0000}, \mi{ANDNML \$X,\haddr{F000}} will result in \haddro{0000}{F0F0}{0F00}{0000}.

\instrtbl
	{\mi{INCH|INCMH|INCML|INCL \$X,YZ}}
	{$\dr{X} \leftarrow (\dr{X} + ({\tt YZ} \ll 48|32|16|0)) \bmod 2^{64}$}
\noindent Last but not least, the {\tt INCX} family adds the 16-bit constant {\tt YZ} to the corresponding wyde of \dr{X}, ignoring overflow \citep[pg. 7]{mmix-doc}. For example, if \dr{X} is \haddro{0000}{F0F0}{FF00}{0000}, \mi{INCML \$X,\haddr{0101}} will result in \haddro{0000}{F0F1}{0001}{0000} (as shown with the example, other wyde may be affected as well, in contrast to the other wyde instructions).

\subsection{Exotic Bit Operations}

Apart from the simple bit operations just described, MMIX does also support more exotic ones that probably will not be used very often, but allow to do complicated computations in hardware instead of in software, as it would be necessary with other architectures.

\instrtbl
	{\mi{MUX \$X,\$Y,\$Z|Z}}
	{$\dr{X} \leftarrow (\dr{Y} \land \sr{M}) \lor (\udrim{Z} \land \mathbin{\overline{\sr{M}}})$}
\noindent The first rather exotic operation is the \i{bitwise multiplexer} \mi{MUX}. For each bit position $i$, it sets bit $\dr{Y}_i$, if $\sr{M}_i$ is 1, and bit $\udrim{Z_i}$, if $\sr{M}_i$ is 0 \citep[pg. 7]{mmix-doc}. For example, if \sr{M} is \haddro{FFFF}{0000}{FFFF}{0000}, \dr{0} is \haddro{1234}{5678}{90AB}{CDEF} and \dr{1} is \haddro{FFFF}{FFFF}{FFFF}{FFFF}, a \mi{MUX \$X,\$0,\$1} will set \dr{X} to \haddro{1234}{FFFF}{90AB}{FFFF}.

\instrtbl
	{\mi{BDIF|WDIF|TDIF|ODIF \$X,\$Y,\$Z|Z}}
	{$\dr{X}_i \leftarrow max(0,\dr{Y}_i - \udrim{Z_i})$ for each byte|wyde|tetra|octa $i$}
\noindent The second family in this category is the byte, wyde, tetra and octa difference. For example, \mi{BDIF} takes the byte $i$ of \dr{Y}, namely $\dr{Y}_i$, and substracts the byte $\udrim{Z_i}$ from it. If the difference is less than zero, $\dr{X}_i$ will be set to zero. Otherwise it will be set to the difference. This is done individually for every byte pair. \mi{WDIF}, \mi{TDIF} and \mi{ODIF} behave analogous using wydes, tetras and octas, respectively. These instructions are for example useful, when a graphical application wants to calculate the "pixel difference", \ie the absolute difference of colors, each color component represented as a byte. \citep[pg. 8]{mmix-doc}

\instrtbl
	{\mi{SADD \$X,\$Y,\$Z|Z}}
	{$\dr{X} \leftarrow countbits(\dr{Y} \setminus \udrim{Z})$}
\noindent The \i{sideways addition} \mi{SADD} performs at first the complement of \udrim{Z} and logically ands the result with \dr{Y}. Afterwards the number of set bits in this value is put into \dr{X}. \citep[pg. 9]{mmix-doc} So, for example, if \dr{0} is \haddr{8642} and \dr{1} is \haddr{8002}, \mi{SADD \$X,\$0,\$1} will put 3 into \dr{X}. Because the difference, \haddr{0640}, has 3 bits set.

\instrtbl
	{\mi{MOR|MXOR \$X,\$Y,\$Z|Z}}
	{$\dr{X} \leftarrow mat(\dr{Y}) \lor|\oplus mat(\udrim{Z})$}
\noindent The last exotic bit operation, called \i{multiple or/exclusive-or}, is the most complicated one. It treats \dr{Y} and \udrim{Z} as $8 \times 8$ bit matrices, using one byte for each column, and performs a kind of matrix product using \mi{OR} or \mi{XOR} instead of the multiplication. More precisely, when the bits of \dr{Y} and \udrim{Z} are numbered as
$$y_{00}y_{01}\ldots y_{07}y_{10}y_{11}\ldots y_{17}\ldots y_{70}y_{71}\ldots y_{77}\quad
z_{00}z_{01}\ldots z_{07}z_{10}z_{11}\ldots z_{17}\ldots z_{70}z_{71}\ldots z_{77},$$
each bit $x_{ij}$ of \dr{X} is set to
$$(y_{0j}\land z_{i0})\lor (y_{1j}\land z_{i1})\lor \cdots \lor (y_{7j}\land z_{i7}).$$
When using \mi{MXOR} instead of \mi{MOR}, the {\tt OR}s are replaced by {\tt XOR}s. \mi{MOR} can be used for example to convert between \glslink{Endianness}{big-endian} and \glslink{Endianness}{little-endian}. \citep[pg. 9]{mmix-doc} If \dr{0} is \haddro{0123}{4567}{89AB}{CDEF} and \dr{1} is \haddro{0102}{0408}{1020}{4080}, a \mi{MOR \$X,\$0,\$1} will perform the following operation:
\[
\left(%
\begin{array}{*{8}{>{\centering\arraybackslash$}p{1mm}<{$}}}
	0 & 0 & 0 & 0 & 1 & 1 & 1 & 1 \\
	\rowcolor{lightgray}0 & 0 & 1 & 1 & 0 & 0 & 1 & 1 \\
	0 & 1 & 0 & 1 & 0 & 1 & 0 & 1 \\
	0 & 0 & 0 & 0 & 0 & 0 & 0 & 0 \\
	0 & 0 & 0 & 0 & 1 & 1 & 1 & 1 \\
	0 & 0 & 1 & 1 & 0 & 0 & 1 & 1 \\
	0 & 1 & 0 & 1 & 0 & 1 & 0 & 1 \\
	1 & 1 & 1 & 1 & 1 & 1 & 1 & 1
\end{array}
\right)
\lor
\left(%
\begin{array}{*{8}{>{\centering\arraybackslash$}p{1mm}<{$}}}
	0 & 0 & 0 & \cellcolor{lightgray}0 & 0 & 0 & 0 & 1 \\
	0 & 0 & 0 & \cellcolor{lightgray}0 & 0 & 0 & 1 & 0 \\
	0 & 0 & 0 & \cellcolor{lightgray}0 & 0 & 1 & 0 & 0 \\
	0 & 0 & 0 & \cellcolor{lightgray}0 & 1 & 0 & 0 & 0 \\
	0 & 0 & 0 & \cellcolor{lightgray}1 & 0 & 0 & 0 & 0 \\
	0 & 0 & 1 & \cellcolor{lightgray}0 & 0 & 0 & 0 & 0 \\
	0 & 1 & 0 & \cellcolor{lightgray}0 & 0 & 0 & 0 & 0 \\
	1 & 0 & 0 & \cellcolor{lightgray}0 & 0 & 0 & 0 & 0
\end{array}
\right)
=
\left(%
\begin{array}{*{8}{>{\centering\arraybackslash$}p{1mm}<{$}}}
	1 & 1 & 1 & 1 & 0 & 0 & 0 & 0 \\
	1 & 1 & 0 & \cellcolor{lightgray}0 & 1 & 1 & 0 & 0 \\
	1 & 0 & 1 & 0 & 1 & 0 & 1 & 0 \\
	0 & 0 & 0 & 0 & 0 & 0 & 0 & 0 \\
	1 & 1 & 1 & 1 & 0 & 0 & 0 & 0 \\
	1 & 1 & 0 & 0 & 1 & 1 & 0 & 0 \\
	1 & 0 & 1 & 0 & 1 & 0 & 1 & 0 \\
	1 & 1 & 1 & 1 & 1 & 1 & 1 & 1
\end{array}
\right)
\]
Analogous to the matrix product, the highlighted cell in the result matrix is built by {\tt AND}ing each cell in the highlighted row of \dr{Y} with the corresponding highlighted cell of \udrim{Z} individually, starting on the left and top, respectively, and performing an {\tt OR} of all these values. In this case, there is no pair of bits in which both are 1 and thus, the highlighted cell in the result is 0. Doing that for all cells leads to the value \haddro{EFCD}{AB89}{6745}{2301}, \ie the bytes of \dr{0} in the opposite order. \citep[pg. 192]{mmix-buch}

