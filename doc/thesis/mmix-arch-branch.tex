\section{Branches and Jumps}

Of course, MMIX does also need instructions to change the course of computation. To allow programs to use a pipeline implementation of MMIX in an efficient way, it provides both ordinary branches and probable branches. For consistency, the different kinds of comparisons offered by branches are the same as those existing for the conditional set and zero or set instructions.

Similarly to the fact, that the typical "set register {\tt X} to the result of {\tt Y} OP {\tt Z}" instructions come in two versions -- one with {\tt Z} as a register, one with {\tt Z} as an \glslink{Immediate Value}{immediate value} -- the branch and jump instructions also come in two versions. The first one branches or jumps forward, while the second one branches or jumps backwards. The backward versions are suffixed with a '{\tt B}'.

\instrtbl
	{\mi{BN|BZ|BP|BOD|BNN|BNZ|BNP|BEV \$X,@+4*(YZ[-$2^{16}$])}}
	{if $s(\dr{X}) <0|=0|>0|odd|\ge0|\ne0|\le0|even$: $@ \leftarrow @+4*({\tt YZ}[-2^{16}])$}
\noindent If \dr{X} is negative, zero, positive, odd, nonnegative, nonzero, nonpositive or even, the branch will be taken. The forward versions increase the \glslink{PC}{instruction pointer} by $4*{\tt YZ}$, \ie the value of the unsigned 16-bit \glslink{Immediate Value}{immediate value} {\tt YZ} multiplied with the number of bytes of an instruction. The backward versions increase it by $4*({\tt YZ}-2^{16})$. Thus, these instructions allow to change $@$ to any instruction in $(@ - 4*2^{16}) \dots (@ + 4*(2^{16}-1))$. It should be noted, that this category of branches tell MMIX that the branch will probably not be taken. This may affect the runtime on some implementations of MMIX. \citep[pg. 12]{mmix-doc}

\instrtbl
	{\mi{PBN|PBZ|PBP|PBOD|PBNN|PBNZ|PBNP|PBEV \$X,@+4*(YZ[-$2^{16}$])}}
	{if $s(\dr{X}) <0|=0|>0|odd|\ge0|\ne0|\le0|even$: $@ \leftarrow @+4*({\tt YZ}[-2^{16}])$}
\noindent These instructions behave exactly in the same way as the previously introduced ones. The only difference is, that these tell MMIX that the branch will probably be taken. \citep[pg. 12]{mmix-doc}

\instrtbl
	{\mi{JMP @+4*(XYZ[-$2^{24}$])}}
	{$@ \leftarrow @+4*({\tt XYZ}[-2^{24}])$}
\noindent Of course, MMIX has also an instruction to change the \glslink{PC}{instruction pointer} unconditionally: the jump. It simply sets $@$ to $(@+4*({\tt XYZ}[-2^{24}]))$, \ie the forward version increases $@$ by the unsigned 24-bit constant {\tt XYZ}, multiplied by 4. The backward version substracts $2^{24}$ from {\tt XYZ} before multiplying. Thus, one can jump to any instruction in the range $(@ - 4*2^{24}) \dots (@ + 4*(2^{24}-1))$. \citep[pg. 13]{mmix-doc}

\instrtbl
	{\mi{GO \$X,\$Y,\$Z|Z}}
	{$\dr{X} \leftarrow @+4,\quad @ \leftarrow \dr{Y} + \udrim{Z}$}
\noindent To be able to jump to any location in the virtual address space, MMIX has the instruction \mi{GO}. It simply changes the \glslink{PC}{instruction pointer} to $\dr{Y} + \udrim{Z}$. Additionally, \dr{X} is set to the location that ordinary would have been executed next. That allows using \mi{GO} for a simple type of \glslink{Subroutine linkage}{subroutine linkage} by not overwriting \dr{X} in the subroutine and returning via \mi{GO \$X,\$X,0}. But MMIX provides another mechanism, that is much better suited for that task, because it makes subroutines independent of each other, as will be described later in this chapter. An interesting corner case is, that \mi{GO} permits it to jump to addresses that are not tetra-aligned. MMIX will simply set the desired \glslink{PC}{instruction pointer}, but this is not going to be a problem because when loading \vmem{4}{@}, the least significant 2 bits of $@$ are ignored. \citep[pg. 13]{mmix-doc}

\instrtbl
	{\mi{GETA \$X,@+4*(YZ[-$2^{16}$])}}
	{$\dr{X} \leftarrow @+4*({\tt YZ}[-2^{16}])$}
\noindent This instruction does not change the \glslink{PC}{instruction pointer}, but is related because it builds an absolute address from the \glslink{PC}{instruction pointer} and the {\tt YZ} field. \mi{GETA} comes in two versions for forward and backwards calculations and uses the same rules for that as the branches do. \citep[pg. 13]{mmix-doc}

