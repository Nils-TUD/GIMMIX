\section{Basic Design Decisions}

Before starting with the description of the simulator, a few general design decisions that have been made should be explained.

\subsection{No Pipelining}

At first, it has been decided to abandon \glslink{Pipelining}{pipelining}. Since it is a simulator, \ie implemented in software, and has not the goal to explore the difficulties for a potential hardware \glslink{Pipelining}{pipelining} implementation or similar, using \glslink{Pipelining}{pipelining} would not bring advantages. Without it, the simulator is much simpler and thus the correctness is easier to achieve.

\subsection{Uninterruptible Instructions}

Similarly to the previous one and a bit related to it, it has been decided to make instructions uninterruptible, \ie an instruction is either executed completely or not at all, as far as no \glslink{Exception}{program or machine exception} occurs. As mentioned in \ref{sec:mmix-interruptibility} of chapter \ref{ch:mmix-arch}, MMIX allows it to make complex instructions like \mi{POP} or \mi{SAVE} interruptible by encoding the current state of computation in \sr{XX}, \sr{YY} and \sr{ZZ}. But it does not require implementations to do that, which means that it is specification conform to make all instructions uninterruptible. \citep[pg. 27]{mmix-doc}

Without this decision, GIMMIX would have required a concept that allows it to execute instructions in multiple steps, pause them for \glslink{Interrupt}{interrupts} and resume them later. This would have made it a lot more complex, without leading to real advantages for similar reasons as in the previous decision. Thus, for simplicity all instructions in GIMMIX are uninterruptible.

\subsection{Programming Language}

Obviously, when one wants to run an operating system in a simulator, \ie a program that implements the whole simulated machine in software, efficiency is very important. Additionally, a lot of control over the produced machine code is necessary to be able to imitate the exact behaviour of the simulated machine. Therefore it has been decided to use the programming language C, which both allows to build efficient programs and offers a lot of control. Additionally, MMIX-SIM and MMIX-PIPE are implemented in C as well\footnote{As mentioned, they are written with CWEB, which produces C code in the end.}, so that it is easier to use code from them, such as the floating point algorithms.

Since an octa is the word quantity of MMIX, it will be used at nearly all places in the simulator. Therefore, its representation affects more or less the whole code. Using \glslink{C89C99}{C89} on a 32-bit platform would mean, that no 64-bit type is available \citep{llgccext}. Thus, a struct would have to be created, that contains two 32-bit integers; one for the upper half of the octa and one for the lower half. Of course, every operation that is done with an octa, would have to be broken down to operations with the two 32-bit integers. In other words, the whole code of the simulator would get a lot more difficult to read and write, just because of the fact that there is no 64-bit type.

But there is an alternative. \glslink{C89C99}{C99} provides the so called \i{exact-width integer type} `uint64_t`, which does always correspond to an unsigned 64-bit integer, independent of the underlying platform \citep{uint64t}. That is, the compiler will arrange things, such that this type (including all possible operations with it) is available. Although most currently available compilers offer no complete implementation of \glslink{C89C99}{C99} yet \citep{c99supp}, even the old version 3.0 (released 2001 \citep{gccrel}) of the GNU C Compiler, which is used in this project, supports this feature including many other useful ones \citep{gcc30c99}. For these reasons, \glslink{C89C99}{C99} has been selected, but GIMMIX will only utilize some basic features of it such as exact-width integer types, "`//` comments", mixed declarations and code, initialization in for-loops and `snprintf`, which are available in \gls{gcc} and should be in almost all other compilers as well.

\subsection{Host Platform}

The host platform, \ie the platform that runs the simulator, is expected to be a Linux system. Although most parts of GIMMIX are platform independent, some of them -- like the simulated terminals for example -- assume Linux. Furthermore, the simulator core (excluding \eg some test generation programs, that expect x86) uses only standard \glslink{C89C99}{C99}, which means it should be hardware independent (but it has not been tested with different hardware).

Since both 32 and 64-bit platforms are widely spread nowadays, GIMMIX has the effort to support both of them. This is mostly achieved behind the scenes by adding a layer on top of the `printf`, `scanf` and `strtol` families. For the first two families, the layer lets the user specify the size of an argument with 'O','T','W' or 'B' (additionally to 'h', 'l' and 'L') for octa, tetra, wyde or byte, respectively. For `strtol`, the layer calls simply either `strtol` or `strtoll`, depending on the platform. But there are a few other things to take care of. For example, when using logical or arithmetical negation, the result depends on the size of the operand. That means, \eg `-sizeof(octa)` could lead to different results. If the type of `sizeof(octa)` is 32 bit wyde, the logical negation would produce \haddrt{FFFF}{FFF8}, instead of the perhaps intended \haddro{FFFF}{FFFF}{FFFF}{FFF8}. Thus, whenever such operators are used with this intention, the operand has to be casted to an octa first.

